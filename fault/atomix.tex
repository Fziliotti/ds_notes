\section{Estudo de caso: Atomix Copycat}
\begin{frame}{Atomix Copycat}
\begin{itemize}
\item Framework de replicação de máquinas de estados implementada pela Atomix. 
\item Implementação do Raft
\item API simples
\item Java 8 (lambdas e futures)
\item \url{http://atomix.io/copycat/}
\end{itemize}
\end{frame}

\begin{frame}[fragile,allowframebreaks]{Lambda}
\begin{itemize}
	\item Classe com um único método.
	\begin{lstlisting}[language=java]
class Tarefa implements Runnable {
  public void run(){
    while (true)
      System.out.println("Bem vindo a um loop infinito");
    }   
}

new Thread(new Tarefa()).start();
	\end{lstlisting}

\framebreak

	\item Classe anônima -- uso único
\begin{lstlisting}[language=java]
new Thread( new Runnable() {
  public void run(){
    while (true)
      System.out.println("Bem vindo a um loop infinito");
  }   
}).start();
\end{lstlisting}

\framebreak

	\item Lambda
\begin{lstlisting}[language=java]
new Thread(() -> {
                     while (true)
                       System.out.println("Bem vindo a um loop infinito");
                   }).start();
\end{lstlisting}

\framebreak
   \item Encadeamento (fluent)
\begin{lstlisting}[language=java]
   Collection<Pessoa> c = ...;
   c.stream()
    .filter(p -> p.idade > 33)
    .map(Pessoa::sobrenomeNome)//.map(p -> p.sobrenomeNome())
    .forEach(s -> System.out.println(s));
\end{lstlisting}
\end{itemize}
\end{frame}

\begin{frame}[fragile]{Future}
\begin{itemize}
	\item Promessa de computação e resultado.
\begin{lstlisting}[language=java]
ExecutorService executor = Executors.newSingleThreadExecutor();
Future<Integer> futFib = executor.submit(() -> { return Fibonacci(217)};
\end{lstlisting}

	\item Quando será executado? \pause Em algum momento.
	\item Como pegar o resultado? 

\begin{lstlisting}[language=java]
while (!futFib.isDone())
  System.out.println("tah calculando...");

int fib217 = futFib.get();
\end{lstlisting}

	\item Em qual thread? \pause Em algum thread. Depende do Executor Service usado.	
\end{itemize}
\end{frame}


\begin{frame}{Atomix-Raft}
\begin{itemize}
	\item Versão >= 2 do copycat
	\item Melhor desempenho
	\item Documentação ruim
	\item \url{https://github.com/atomix/atomix}
\end{itemize}
\end{frame}

\begin{frame}{Lab}
\begin{itemize}
	\item Versão 1.1.4
	\item Baseado em \url{http://atomix.io/copycat/docs/getting-started/}
	\item Código funcional em \url{https://github.com/pluxos/atomix_labs}
\end{itemize}
\end{frame}

\begin{frame}{Clone e compile o projeto}
\begin{itemize}
	\item Instale dependências: git, maven e JDK >= 1.8 (lembre-se que gRPC precisa de JDK <= 1.8)
	\item git clone https://github.com/pluxos/atomix\_labs
	%https://www.baeldung.com/atomix
	\item cd atomix\_labs
	\item cd replication
	\item mvn compile
	\item mvn test
\end{itemize}
\end{frame}

\begin{frame}[fragile]{mvn test}
Resultado esperado.
\begin{verbatim}
Tests run: 1, Failures: 0, Errors: 0, Skipped: 0

[INFO] ---------------------------------------
[INFO] BUILD SUCCESS
[INFO] ---------------------------------------
[INFO] Total time: 6.898 s
[INFO] Finished at: 2017-10-25T08:38:08-02:00
[INFO] Final Memory: 15M/159M
[INFO] ---------------------------------------
\end{verbatim}
\end{frame}

\begin{frame}{Estrutura}
Explore o projeto. Na pasta/URL \url{https://github.com/pluxos/atomix_labs/tree/master/replication/src/main/java/atomix_lab/state_machine}

Há três pastas. Analise-as nesta ordem
\begin{itemize}
	\item type -- tipos dos dados mantidos pela replica (Edge e Vertex)\\
	Os tipos são serializable para que o Java saiba como transformá-los em bytes.
	\item command -- estruturas que contêm informações para modificar os tipos\\
	Os comandos serão enviadas do cliente para o cluster e são naturalmente serializable.
	\item client -- cria comandos e os envia para serem executados no cluster\\
	Respostas podem ser esperadas síncrona ou assincronamente.
	\item server -- recebe os comandos na ordem definida pelo Raft e os executa
\end{itemize}
\end{frame}


\begin{frame}{Lab}
O projeto foi construído seguindo as instruções no tutorial mencionado antes, saltando-se a parte dos snapshots, isto é:
\begin{itemize}
	\item crie um projeto maven\\
	eclipse tem template para isso
	\item adicione dependências no pom.xml\\
	como so criei um projeto, coloquei as dependências tanto do cliente quando do servidor
	\item defina Command que modifiquem o estado das réplicas
	\item defina Queries que consultem o estado das réplicas
	\item implemente a réplica para lidar com os comandos
	\item implemente o cliente para emitir comandos
\end{itemize}
\end{frame}

\begin{frame}{Lab}
Para executar um servidor, você precisa passar como parâmetro
\begin{itemize}
	\item identificador do processo (inteiro)
	\item IP do processo com identificador 0
	\item porta do processo com identificar 0
	\item IP do processo com identificador 1
	\item porta do processo com identificar 1
	\item ...
\end{itemize}

Sabendo seu identificador, o servidor sabe em qual porta escutar e em quais IP/porta se conectar.
\end{frame}

\begin{frame}[fragile]{Lab}
Execute três servidores. Usando o maven, da linha de comando, fica assim:
\begin{tiny}
\begin{verbatim}
mvn exec:java \\
  -Dexec.mainClass="atomix_lab.state_machine.server.GraphStateMachine" \\
  -Dexec.args="0 127.0.0.1 5000 127.0.0.1 5001 127.0.0.1 5002"
	
mvn exec:java \\
  -Dexec.mainClass="atomix_lab.state_machine.server.GraphStateMachine" \\
  -Dexec.args="1 127.0.0.1 5000 127.0.0.1 5001 127.0.0.1 5002"
	
mvn exec:java \\
  -Dexec.mainClass="atomix_lab.state_machine.server.GraphStateMachine" \\
  -Dexec.args="2 127.0.0.1 5000 127.0.0.1 5001 127.0.0.1 5002"
\end{verbatim}
\end{tiny}
%As instruções, sem quebra, estão no README do repositório.
\end{frame}


\begin{frame}[fragile]{Lab}
O cliente não precisa de um identificador, apenas dos pares IP/porta dos servidores.
\begin{itemize}
	\item IP do processo com identificador 0
	\item porta do processo com identificar 0
	\item IP do processo com identificador 1
	\item porta do processo com identificar 1
	\item ...
\end{itemize}

Para executá-lo, use o comando
\begin{tiny}
\begin{verbatim}
mvn exec:java 
  -Dexec.mainClass="atomix_lab.state_machine.client.GraphClient"
  -Dexec.args="127.0.0.1 5000 127.0.0.1 5001 127.0.0.1 5002"
\end{verbatim}
\end{tiny}

\end{frame}

\begin{frame}{Exercício}
Uma vez executado o projeto, modifique-o para incluir uma nova operação (Command) e nova consulta (Query). 
\end{frame}
