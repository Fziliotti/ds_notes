\section{Serviços de Coordenação}

\begin{frame}{Serviços de Coordenação}
\begin{itemize}
	\item Zookeeper
	\item Atomix
	\item OpenReplica
\end{itemize}
\end{frame}

\subsection{Estudo de Caso: ZooKeeper}

\subsubsection{Visão Geral}
\begin{frame}{ZooKeeper}
\begin{center}
\includegraphics{images/zklogo}
\end{center}
\url{http://zookeeper.apache.org/}
\end{frame}


\begin{frame}{ZooKeeper}
\begin{block}{Zoo?}
Porquê sistemas distribuídos são como zoológicos, com animais de diversas espécies, sendo obrigados a conviver de forma anti-natural.
\end{block}
\end{frame}

\begin{frame}{ZooKeeper}
\begin{block}{O quê?}
ZooKeeper is a \alert{centralized} service for maintaining \alert{configuration} information, \alert{naming}, providing distributed \alert{synchronization}, and providing \alert{group services}. All of these kinds of services are used in some form or another by \alert{distributed applications}. Each time they are implemented there is a lot of work that goes into fixing the bugs and race conditions that are inevitable. Because of the difficulty of implementing these kinds of services, applications initially usually skimp on them, which make them brittle in the presence of change and difficult to manage. Even when done correctly, different implementations of these services lead to management \alert{complexity} when the applications are deployed.
\end{block}
\end{frame}


\begin{frame}{ZooKeeper}
\begin{block}{O quê?}
	ZooKeeper is a \alert{distributed}, open-source \alert{coordination service for distributed applications}. It exposes a \alert{simple set of primitives} that distributed applications can build upon to implement higher level services for synchronization, configuration maintenance, and groups and naming. It is designed to be easy to program to, and uses a data model styled after the familiar \alert{directory tree structure of file systems}. It runs in Java and has bindings for both \alert{Java} and \alert{C}.
\end{block}
\end{frame}

\begin{frame}{ZooKeeper}
\begin{block}{Por quê?}
	Coordination services are notoriously hard to get right. They are especially prone to errors such as race conditions and deadlock. The motivation behind ZooKeeper is to relieve distributed applications the responsibility of implementing coordination services from scratch.
\end{block}
\end{frame}


\begin{frame}{ZooKeeper}
\begin{block}{Como?}
	ZooKeeper allows distributed processes to coordinate with each other through a \alert{shared hierarchal namespace which is organized similarly to a standard file system}. The name space consists of data registers - called \alert{znodes}, in ZooKeeper parlance - and these are \alert{similar to files and directories}. Unlike a typical file system, which is designed for storage, ZooKeeper data is kept \alert{in-memory}, which means ZooKeeper can achieve \alert{high throughput and low latency} numbers.
\end{block}

\end{frame}

\begin{frame}{ZooKeeper}
\includegraphics[width=.7\textwidth]{images/zknamespace}
\end{frame}

\begin{frame}{ZooKeeper}
\begin{block}{Como?}
	ZooKeeper is replicated.\\
	ZooKeeper is ordered.
\end{block}

\includegraphics[width=1\textwidth]{images/zkservice}
\end{frame}

\begin{frame}{ZooKeeper}
\includegraphics[width=1\textwidth]{images/zkcomponents}
\end{frame}

\begin{frame}{ZooKeeper}
\begin{block}{Como?}
	ZooKeeper is fast [...] and it performs best where reads are more common than writes, at ratios of around 10:1.
\end{block}
\end{frame}

\begin{frame}{Desempenho}
	\includegraphics[width=1\textwidth]{images/zkperfRW_3_2}
\end{frame}


\subsubsection{Uso}
\begin{frame}{ZNodes}
\includegraphics[width=.6\textwidth]{images/zknamespace}

\begin{itemize}
	\item Arquivo e diretório ao mesmo tempo.
	\item São (devem ser) pequenos.
	\item Operados atomicamente: todo o dado é lido/escrito.
	\item API simples
		\begin{itemize}
			\item C: create
			\item R: get
			\item U: set
			\item D: delete
			\item *: get children
		\end{itemize}
\end{itemize}
\end{frame}


\begin{frame}[fragile]{Lab}
\begin{itemize}
	\item Download: wget \url{www-eu.apache.org/dist/zookeeper/zookeeper-3.4.10}
	\item Unpack: tar xvzf zookeeper*.tgz
	\item Config: \verb|conf/zoo.cfg| $\Leftarrow$ Copie o exemplo\\
	%\verb|tickTime=2000|\\
	%\verb|dataDir=/tmp/seuNome/zk0| $\Leftarrow$\\
	%\verb|clientPort=2181| $\Leftarrow$
	\item \verb|./bin/zkServer.sh start-foreground|
	\item \verb|./bin/zkCli.sh -server 127.0.0.1:2181| $\Leftarrow$
\end{itemize}
\end{frame}


\frame{Exemplo}

\begin{frame}{ZNodes}
\begin{itemize}
	\item Stat(istics): versão, ACL, timestamps.
\end{itemize}
\end{frame}

\frame{Exemplo}

\begin{frame}{ZNodes}
\begin{itemize}
	\item Updates condicionais.
\end{itemize}
\end{frame}

\frame{Exemplo}

\begin{frame}{ZNodes}
\begin{itemize}
	\item Nós efêmeros: presentes enquanto a sessão que os criou estiver ativa.
\end{itemize}
\end{frame}

\frame{Exemplo}

\begin{frame}{ZNodes}
\begin{itemize}
	\item Watches: notificam clientes de mudanças no nó ou em seus filhos.
	\item Uso único.
\end{itemize}
\end{frame}

\frame{Exemplo}

\begin{frame}{Durabilidade?}
Todas as operações são colocadas em um log em disco.
\end{frame}


\subsubsection{Receitas}
\begin{frame}{Receitas}
É possível resolver diversos problemas encontrados em sistemas distribuídos usando-se o ZooKeeper.
\end{frame}

\begin{frame}{Rendezvous}
Ponto de encontro de processos. \pause

\begin{itemize}
	\item Defina um zNode raiz a ser usado: /rendezvous/app1/ \pause
	\item Cada filho de /rendezvous/app1 corresponde a um processo:
		\begin{itemize}
			\item IP
			\item Porta
			\item Número de processadores
			\item ...
		\end{itemize}
	\item Processo p ao ser iniciado:
		\begin{itemize}
			\item procura /rendezvous/app1/p
			\begin{itemize}
				\item se achar, continua
				\item se não achar, cria /rendezvous/app1/p
			\end{itemize}
			\item lista os filhos de /rendezvous/app1
		\end{itemize}
\end{itemize}
\end{frame}

\begin{frame}{Como lidar com saída de processos?}
\pause Faça todos os zNodes são efêmeros. \\
Quando um nó é desconectado, o zNode correspondente será destruído.
\end{frame}

\begin{frame}{Como detectar mudanças no grupo de processos?}
Monitore os filhos de /rendezvous/app1\\
Sempre que receber notificações, refaça o cálculo do \emph{membership}.
\end{frame}

\begin{frame}{Eleição de Líderes}
\pause Rendezvous.\\
\pause Faça os zNodes sequenciais.
\pause Ordene os zNodes e escolha o primeiro.
\pause Monitore o zNode. Se ele sumir, eleja outro líder.
\end{frame}

\begin{frame}{Exclusão Mútua}
Construa uma fila usando nós efêmeros e sequenciais. O processo na cabeça da fila tem direito de acesso. Em caso de falhas, o processo é removido da cabeça da fila.
\end{frame}

\begin{frame}{Compartilhamento de Parâmetros de Configuração}
Pronto!
\end{frame}

\begin{frame}{Receitas}
\begin{itemize}
	\item Lock distribuído
	\item Filas, e.g. de prioridades
	\item Barreira
	\item Serviço de nomes
	\item Terminação em duas fases
	\item Contador atômico
\end{itemize}

\url{http://zookeeper.apache.org/doc/trunk/recipes.html}
\end{frame}

\begin{frame}{Curator}
\includegraphics{images/curator-logo}

Um livro de receitas implementadas em ZK.

\url{http://curator.apache.org}
\end{frame}

\begin{frame}{Lab}
	\begin{itemize}
		\item Crie um zNode /teste
		\item Debaixo de /teste, crie três outros, sequenciais
	\end{itemize}
\end{frame}

\begin{frame}{Lab}
\begin{itemize}
	\item Crie um zNode /teste2
	\item Crie um zNode efêmero
	\item Conecte-se com outro cliente
	\item Coloque um watch em /teste2
	\item Desconecte o primeiro cliente
	\item Observe o evento gerado no segundo cliente
	\item Reconecte o primeiro cliente
\end{itemize}
\end{frame}

\begin{frame}[fragile]{Multi-node}
Crie três arquivos, zoo1.cfg, zoo2.cfg e zoo3.cfg.\\Por exemplo, zoo1.cfg fica assim:
\begin{itemize}
	\item \verb|dataDir=/tmp/lasaro/zoo1| $\Leftarrow$ Diretórios distintos.
	\item \verb|server.1=zoo1:2888:3888| $\Leftarrow$ Portas distintas.
	\item \verb|server.2=zoo2:2889:3889|
	\item \verb|server.3=zoo3:2890:3890|
	\item \verb|clientPort=2181| $\Leftarrow$ Portas distintas.
	
\end{itemize}

Crie diretórios e arquivos de identificação.
\begin{itemize}
	\item \verb|mkdir /tmp/lasaro/zoo1|
	\item \verb|echo 1 > /tmp/lasaro/zoo1/myid|
\end{itemize}

Execute servidores.
\begin{itemize}
	\item \verb|./bin/zkServer.sh start conf/zoo1.cfg|
\end{itemize}

Use \verb|start-foreground| para acompanhar a execução.

\end{frame}

\begin{frame}{Lab}
\begin{itemize}
	\item Crie um znode /contador com valor 0
	\item Descreva como fazer para que os clientes incrementem atomicamente o valor de /contador.
\end{itemize}
\end{frame}